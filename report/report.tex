\documentclass{article}


\usepackage{graphicx}
\usepackage{listings}
\usepackage{color}

\definecolor{dkgreen}{rgb}{0,0.6,0}
\definecolor{gray}{rgb}{0.5,0.5,0.5}
\definecolor{mauve}{rgb}{0.58,0,0.82}

\lstset{frame=tb,
  language=C++,
  aboveskip=3mm,
  belowskip=3mm,
  showstringspaces=false,
  columns=flexible,
  basicstyle={\small\ttfamily},
  numbers=none,
  numberstyle=\tiny\color{gray},
  keywordstyle=\color{blue},
  commentstyle=\color{dkgreen},
  stringstyle=\color{mauve},
  breaklines=true,
  breakatwhitespace=true,
  tabsize=3
}

\begin{document}

    \title{%
      Metaballs in GLFW \\
      \large DD2323 Project Report \\}

    \author{Alexander Hjelm (alhjelm@kth.se), Tsz Kin Chan (XXXXXXXXXXXXXXXXx@kth.se)}

    \date{\today}

    \maketitle


    \section{Summary}
    Hippety hoppety, Women are property! >:)
    
    \section{Introduction}
    

    \section{Theory}

    \subsection{Metaball isosurface model}
    Potential field where each ball has an influence point with quadratic falloff.

    \subsection{Marching cubes}
    Voxel shader with an array of predefined cube shapes. The shader selects the relevant cube shape depending on how the neighbourhood looks with respect to the potential.
    
    \subsection{Misc}
    Additionally we implemented a fragment shader with basic Phong illumination, and added a physics model to the balls.

    \section{Implementation}
        \begin{lstlisting}
        private void Test(){
            int a = 1;
        }
        \end{lstlisting}

    \section{Result}
    *Screenshot*
    *Link to an animated video*

    \section{Conslusions}
    
    Smoothed Particle Hydrodynamics
    Ellipsoid Splatting

    The voxel grid size is the performance bottleneck.
    (Possible conslusion: spatial subdivision of metaball positions will not yield the neccessary performance increase)

\end{document}
